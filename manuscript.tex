\title{An Adaptive Branch Site REL Model for Efficient Detection of Episodic
Diversifying Selection}

\author{Martin D Smith, Joel Wertheim, Sergei L Kosakovsky Pond\\
        Bioinformatics And Systems Biology\\
        University of California, San Diego\\\\
}
\date{\today}

\documentclass[12pt]{article}
\usepackage{tabularx}
\usepackage{url}
\usepackage{outlines}
\usepackage{enumitem}
\usepackage{fullpage}
\usepackage{graphicx}
\usepackage{wrapfig}
\usepackage[footnotesize]{caption}
\setcounter{secnumdepth}{5}
\setlength{\captionmargin}{48pt}
\captionsetup[figure]{labelfont=bf}
%\DeclareGraphicsRule{.tif}{png}{.png}{`convert #1 `dirname #1`/`basename #1 .tif`.png}

\begin{document}
\begin{titlepage}
\maketitle
\thispagestyle{empty}

\begin{abstract}
Models of sequence evolution have become an important part of understanding a
wide variety of species and a fundamental part of the battle against HIV and
other epidemics. Their application to viral evolution is not exclusive to
vaccine design, as questions about the origin of zoonoses, for instance,
become increasingly interesting to epidemiology and other fields. Due in
part to their computational complexity, codon models of sequence evolution
are still less frequently used for selection detection than the often
insufficient GTR+$\gamma$ nucleotide model. Codon models that allow for
heterogeneity in substitution rates across sites, branches, or both, when
freed from \textit{a priori} assumptions about the appropriate class of each
brance, have been shown to detect positive selection more effectivly than
even the next most advanced heterogeneous codon models. Previously published
Branch Site REL models used a fixed number of rate classes per branch,
leading to higher-than-necessary computational complexity and occassional
numerical instability. Here we update this model to introduce a model-testing
based Branch Site REL model with an adaptive number of rate classes per
branch, achieving the same or better power and sensitivity, increased
performance and increased numerical stability.
\end{abstract}
\end{titlepage}

\pagebreak

\section{Introduction}
Lorem ipsum dolor sit amet, consectetur adipisicing elit, sed do eiusmod
tempor incididunt ut labore et dolore magna aliqua. Ut enim ad minim veniam,
quis nostrud exercitation ullamco laboris nisi ut aliquip ex ea commodo
consequat. Duis aute irure dolor in reprehenderit in voluptate velit esse
cillum dolore eu fugiat nulla pariatur. Excepteur sint occaecat cupidatat non
proident, sunt in culpa qui officia deserunt mollit anim id est laborum.

\bibliographystyle{abbrv}
\bibliography{Writeup}

\end{document}
